% Options for packages loaded elsewhere
\PassOptionsToPackage{unicode}{hyperref}
\PassOptionsToPackage{hyphens}{url}
%
\documentclass[
]{book}
\usepackage{amsmath,amssymb}
\usepackage{lmodern}
\usepackage{iftex}
\ifPDFTeX
  \usepackage[T1]{fontenc}
  \usepackage[utf8]{inputenc}
  \usepackage{textcomp} % provide euro and other symbols
\else % if luatex or xetex
  \usepackage{unicode-math}
  \defaultfontfeatures{Scale=MatchLowercase}
  \defaultfontfeatures[\rmfamily]{Ligatures=TeX,Scale=1}
\fi
% Use upquote if available, for straight quotes in verbatim environments
\IfFileExists{upquote.sty}{\usepackage{upquote}}{}
\IfFileExists{microtype.sty}{% use microtype if available
  \usepackage[]{microtype}
  \UseMicrotypeSet[protrusion]{basicmath} % disable protrusion for tt fonts
}{}
\makeatletter
\@ifundefined{KOMAClassName}{% if non-KOMA class
  \IfFileExists{parskip.sty}{%
    \usepackage{parskip}
  }{% else
    \setlength{\parindent}{0pt}
    \setlength{\parskip}{6pt plus 2pt minus 1pt}}
}{% if KOMA class
  \KOMAoptions{parskip=half}}
\makeatother
\usepackage{xcolor}
\usepackage{longtable,booktabs,array}
\usepackage{calc} % for calculating minipage widths
% Correct order of tables after \paragraph or \subparagraph
\usepackage{etoolbox}
\makeatletter
\patchcmd\longtable{\par}{\if@noskipsec\mbox{}\fi\par}{}{}
\makeatother
% Allow footnotes in longtable head/foot
\IfFileExists{footnotehyper.sty}{\usepackage{footnotehyper}}{\usepackage{footnote}}
\makesavenoteenv{longtable}
\usepackage{graphicx}
\makeatletter
\def\maxwidth{\ifdim\Gin@nat@width>\linewidth\linewidth\else\Gin@nat@width\fi}
\def\maxheight{\ifdim\Gin@nat@height>\textheight\textheight\else\Gin@nat@height\fi}
\makeatother
% Scale images if necessary, so that they will not overflow the page
% margins by default, and it is still possible to overwrite the defaults
% using explicit options in \includegraphics[width, height, ...]{}
\setkeys{Gin}{width=\maxwidth,height=\maxheight,keepaspectratio}
% Set default figure placement to htbp
\makeatletter
\def\fps@figure{htbp}
\makeatother
\setlength{\emergencystretch}{3em} % prevent overfull lines
\providecommand{\tightlist}{%
  \setlength{\itemsep}{0pt}\setlength{\parskip}{0pt}}
\setcounter{secnumdepth}{5}
\usepackage{booktabs}
\usepackage{amsthm}
\makeatletter
\def\thm@space@setup{%
  \thm@preskip=8pt plus 2pt minus 4pt
  \thm@postskip=\thm@preskip
}
\makeatother
\ifLuaTeX
  \usepackage{selnolig}  % disable illegal ligatures
\fi
\usepackage[]{natbib}
\bibliographystyle{apalike}
\IfFileExists{bookmark.sty}{\usepackage{bookmark}}{\usepackage{hyperref}}
\IfFileExists{xurl.sty}{\usepackage{xurl}}{} % add URL line breaks if available
\urlstyle{same} % disable monospaced font for URLs
\hypersetup{
  pdftitle={AI and the Evolving Competitive Landscape: Navigating Regulatory and Strategic Shifts},
  pdfauthor={Manav Adwani, Sohil Apte, Lauren Brown, Anson Lee, Xuexin Li},
  hidelinks,
  pdfcreator={LaTeX via pandoc}}

\title{AI and the Evolving Competitive Landscape: Navigating Regulatory and Strategic Shifts}
\author{Manav Adwani, Sohil Apte, Lauren Brown, Anson Lee, Xuexin Li}
\date{2023-08-19}

\begin{document}
\maketitle

{
\setcounter{tocdepth}{1}
\tableofcontents
}
\hypertarget{introduction}{%
\chapter{Introduction}\label{introduction}}

From assisting with strategy to generating content, artificial intelligence (AI) is already influencing how businesses operate. The consulting firm Accenture projects an average profitability boost of 38\% by 2035 across sixteen major industries. Additionally, AI has led to increased productivity and reduced costs, particularly in analytics, product development, and customer service. This has led many businesses to adopt AI with large firms like Google and Microsoft potentially investing billions in the technology. It is apparent that AI will provide a new arena for businesses to compete in.

Yet, the excitement over the benefits of AI has yielded challenges for businesses and regulators. For businesses, there are questions on what duties to assign AI. One of the causes of the ongoing actor and writer strikes is concerns over AI replacing human workers. Technology contractors in the Republic of Kenya are suing Meta for being exposed to psychologically damaging content while moderating its generative AI systems. Additionally, the exact information necessary for AI to arrive at optimal solutions is not known. Recent studies have shown there are limits to the effectiveness of pricing algorithms and that data reaches a point of diminishing returns. AI has also proven to be difficult to regulate; governments and institutions face the challenge of controlling something without knowing its full potential. The laws they establish will inevitably provide another dimension to business competition.

As businesses adopt AI, it is crucial to consider both the benefits and challenges to best predict its impact on the competitive landscape. To what degree will businesses rely on these systems and how will their competitors respond? What industries are likely to see the most meaningful change in operations? And what does this mean for the workers within those industries? Acknowledging these questions is the key to developing a business environment that utilizes AI in an efficient, fair, and ethical manner.

\hypertarget{about-us}{%
\chapter{About Us}\label{about-us}}

\hypertarget{mbangels}{%
\section{MBAngels}\label{mbangels}}

We are a team of Master of Business Analytics (MBAn) students with the goal of understanding how the new age of generative artificial intelligence (AI) will impact the world, and specifcally, impact our cohort as analytics students. By combining our wide array of backrounds, diverse skillsets, and range of persepctives, we hope to compile a robust report that captures the impact AI will have on all levels of industry.

\hypertarget{manav-adwani}{%
\section{Manav Adwani}\label{manav-adwani}}

\textbf{Background:}
Manav studied commerce in his undergrad and comes from a family business background. After completing a relatively generic degree for his undergrad, he now wants to explore the true power of business when data analytics is applied to it. This would include learning more about the strategy side of analytics.

\textbf{Experience:}
While he is enthusiastic about learning the technical aspect, he hopes to provide an in depth business perspective to the team. The team can benefit from his experience in EY and his knowledge about setting up and running a successful international business as an entrepreneur.

\hypertarget{sohil-apte}{%
\section{Sohil Apte}\label{sohil-apte}}

\textbf{Background:}
Sohil is originally from Canton Ma, and graduated from the University of Michigan in 2023. With a BS in Computer Science and a concentration on artificial intelligence and machine learning, Sohil has a robust theoretical perspective of AI, and an understanding of the intricacies of machine learning applications.

\textbf{Experience:}
As a former Computer Science student with practical experience as a Software/Machine Learning Engineer at a startup, his knowledge of machine learning model development and deployment, along with its business implications, offers insights into the real-world competitive impact of AI technologies.

\hypertarget{lauren-brown}{%
\section{Lauren Brown}\label{lauren-brown}}

\textbf{Background:}
Lauren is from Brighton, Michigan and graduated from the school of LSA at the University of Michigan in 2023. Being a Psychology major, Lauren is interested in society's reaction to AI and its outcomes.

\textbf{Experience:}
Lauren has experience in marketing, and has developed a passion for AI regulation as she recognized the potential for AI's powerful impact on the industry alongside the need for ethical guidelines to protect consumers and ensure fair practices.

\hypertarget{anson-lee}{%
\section{Anson Lee}\label{anson-lee}}

\textbf{Background:}
Anson Lee is originally from Poughkeepsie, New York. He graduated with a BBA from the Ross School of Business in 2023.

\textbf{Experience:}
His experience includes working in the operations departments of Goldman Sachs and Toyoda Gosei North America. He is also an avid cinephile and skier. Anson Lee is primarily concerned with how AI will be regulated to protect workers and industries.

\hypertarget{xuexin-li}{%
\section{Xuexin Li}\label{xuexin-li}}

\textbf{Background:}
Xuexin is originally from Beijing, China. Xuexin studied in Business Management and Economics at UC Santa Cruz during her undergraduate years.

\textbf{Experience:} Xuexin has immersed herself in the Digital Marketing industry in China after graduation. Her experience has sparked a deep interest in exploring AI's transformative potential in reshaping marketing strategies and consumer behavior.

Our combined skills and shared passion for the topic make us uniquely equipped to tackle the questions surrounding AI and the impacts it will have on comeptition and strategy. We are excited to share our research and insights in this paper.

\hypertarget{the-rise-of-ai-in-business}{%
\chapter{The Rise of AI in Business}\label{the-rise-of-ai-in-business}}

\hypertarget{xuexin-likes-the-simpsons.}{%
\section{Xuexin likes the Simpsons.}\label{xuexin-likes-the-simpsons.}}

\hypertarget{manavs-backk}{%
\section{Manav's backk!!}\label{manavs-backk}}

Please let this be a formal apology to Sohil and everyone else in the group for not passing it off to him for the Bobby Madamanchi quote in our group presentation.

\hypertarget{the-influence-of-ai-on-business-operations}{%
\section{The Influence of AI on Business Operations}\label{the-influence-of-ai-on-business-operations}}

\hypertarget{the-potential-of-ai-productivity-and-profitability}{%
\section{The Potential of AI: Productivity and Profitability}\label{the-potential-of-ai-productivity-and-profitability}}

\hypertarget{major-players-ai-investments-by-google-and-microsoft}{%
\section{Major Players: AI Investments by Google and Microsoft}\label{major-players-ai-investments-by-google-and-microsoft}}

\hypertarget{ai-and-competition}{%
\chapter{AI and Competition}\label{ai-and-competition}}

\hypertarget{amazon-case-study}{%
\section{Amazon Case Study}\label{amazon-case-study}}

\hypertarget{streaming-service-recommendation}{%
\section{Streaming Service Recommendation}\label{streaming-service-recommendation}}

\hypertarget{ais-impact-on-workers}{%
\chapter{AI's Impact on Workers}\label{ais-impact-on-workers}}

\hypertarget{impact-on-the-ross-student}{%
\section{Impact on the Ross Student}\label{impact-on-the-ross-student}}

\hypertarget{regulation-of-ai}{%
\chapter{Regulation of AI}\label{regulation-of-ai}}

AI technology has been making major advances with AI systems, such as ChatGPT, being used by over 40\% of college students in the United States (Balderson, 2023). Companies and organizations all around the world are increasing their use of AI to gain a competitive advantage in their respective industries. This race of AI technology is putting focus on what regulation needs to be developed to control these advances. AI is becoming a core part of political discourse, and big names in the tech industry such as Steve Wozniak, Elon Musk, and politicians such as Andrew Yang, have signed an open letter asking for a moratorium on AI experimentation for regulation development (Pause Giant AI experiments: An open letter, 2023). While this open letter has not led to any action, it does bring attention to the fact that governments will have to focus a large part of their future legislation on regulation. AI systems and regulations are global and inter-industry issues. There are so many different pieces, in different industries, and they're all moving at different paces, which makes it difficult to legislate.

\hypertarget{current-regulation}{%
\section{Current regulation}\label{current-regulation}}

Governments around the world have only scratched the surface on legislating AI regulation. Most countries, like the United States, are still debating what parts of AI should be regulated, and how much regulation is appropriate. There are a few exceptions to this trend. The European Parliament has recently taken its first step to legal regulation by passing a draft law named the AI Act. This act will focus on limiting the most high-risk parts of AI technology, such as facial recognition. It will also force upcoming AI systems to be more open about the data they are using. While this is only a draft law, the final version of this law is expected to be passed by the end of 2023, making it one of the first large steps toward government-enforced AI regulation (Satariano, 2023).

\hypertarget{proposed-regulation}{%
\section{Proposed regulation}\label{proposed-regulation}}

AI regulations are slowly starting to be put into effect, but there are still so many questions on how it will work. The CEO of OpenAI, Sam Altman, wrote a testimony to the Senate this May asking that they start strongly considering regulating artificial intelligence (Hendrix, 2023). He proposed solutions such as licensing companies to use AI, or creating a government agency strictly responsible for AI regulation. Altman's solutions were questioned by some, though, as there are still many disagreements about what the best way to legislate AI should be. While these solutions seem good on paper, the answer to how to regulate AI is not that simple. Officials are predicting that AI regulation agencies have the potential of being compromised. Having only one small sector of government responsible for such a broad and important issue could lead to bias. Experts have recommended that AI needs to be diversely regulated by a combination of different agencies and organizations. They have said that regulation should be created by a collaboration from academia, policy experts, industry experts, and even international agencies (Susarla, 2023).

New strategies for AI regulation are being proposed regularly by CEOs and other big names in the tech field. AI technology is progressing rapidly and there is an understanding of the need to regulate it before it gets bigger, faster, and stronger. While most tech officials are in agreement about the importance of regulation, government officials can often be harder to come by. The median age of the U.S. The House of Representatives is 58 years old, and the median age is 65 years old in the Senate. The United States lawmaker demographic is much older than the mean age in America (38 years old) (Blazina and DeSilver, 2023). Their age could mean they have lesser knowledge or insight into these new technological advancements than young, up-and-coming workers in the tech industry do. The lawmakers' lack of understanding of artificial intelligence applications can lead to more debate than needed on what type of regulation should be established.

These are prime years to establish effective regulations on artificial intelligence and how it is used. Setting regulations now, when AI is not overwhelmingly powerful, could be key to getting a grasp on this technology and its future that we cannot predict. Some places, like the European Union, are taking their first steps toward government action, but others, like the United States, have barely scratched the surface. While companies are already beginning to integrate AI regulation into their business, not every organization will follow suit without government intervention. AI Regulation should be expected to be a hot topic in government and politics in the near future since AI will continue to progress and the demand for regulation will increase with it.

\hypertarget{regulation-as-a-competitive-advantage}{%
\section{Regulation as a competitive advantage}\label{regulation-as-a-competitive-advantage}}

AI use within companies is becoming increasingly normalized in numerous industries - leading to a higher mistrust in AI and AI regulation among customers and employees of these companies. So much of what artificial intelligence is can be difficult for the average individual to grasp. For some people, the first thing they think of when they hear ``AI'' is horror stories of technology taking over the world. Acknowledging these individuals and understanding their views is a step that some companies have already started to take.

\hypertarget{pros}{%
\subsection{Pros}\label{pros}}

Responsible AI is a process that certain companies have acquired to regulate their own use of AI. It is the process of developing AI systems that minimizes biases, enforces data security, ensures transparency within companies, and creates opportunities for employees to speak out on their concerns on AI use. Companies using this form of AI regulation have actually found that it can be used as a competitive advantage. Currently, 36\% of organizations have said they believe it will create opportunities for competitive differentiation (Eitel-Porter, 2023). The ability to deliver trustworthy AI systems that are regulation-ready can help a company attract new customers because they are using AI safely and are being transparent about its use. While this is a step in the right direction, AI regulation needs to be supported by governments to truly make a difference.

\hypertarget{cons}{%
\subsection{Cons}\label{cons}}

While there are obvious benefits to regulating the use of AI, there are also a number of setbacks that could have a negative effect. Limiting its capabilities or room for growth could lead to artificial intelligence not having the ability or flexibility to adapt or learn in positive ways. Strict regulations could restrain creativity and innovation, and lessen the amount of risks taken by companies. This also could become a problem in relation to international affairs. Companies within countries that set stricter regulations will suffer in comparison to companies rooted in countries that have more flexible AI control (Balderson, 2023).

\hypertarget{potential-future-uses-for-competition-and-future-outlook}{%
\chapter{Potential future uses for competition and future outlook}\label{potential-future-uses-for-competition-and-future-outlook}}

manav manav manav manav manav

\hypertarget{quotes-from-someone-elses-predictions}{%
\section{Quotes From Someone Else's Predictions}\label{quotes-from-someone-elses-predictions}}

\hypertarget{manav-is-late}{%
\section{Manav is late!}\label{manav-is-late}}

\hypertarget{conclusion}{%
\chapter{Conclusion}\label{conclusion}}

  \bibliography{book.bib,packages.bib}

\end{document}
