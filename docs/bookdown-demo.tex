% Options for packages loaded elsewhere
\PassOptionsToPackage{unicode}{hyperref}
\PassOptionsToPackage{hyphens}{url}
%
\documentclass[
]{book}
\usepackage{amsmath,amssymb}
\usepackage{iftex}
\ifPDFTeX
  \usepackage[T1]{fontenc}
  \usepackage[utf8]{inputenc}
  \usepackage{textcomp} % provide euro and other symbols
\else % if luatex or xetex
  \usepackage{unicode-math} % this also loads fontspec
  \defaultfontfeatures{Scale=MatchLowercase}
  \defaultfontfeatures[\rmfamily]{Ligatures=TeX,Scale=1}
\fi
\usepackage{lmodern}
\ifPDFTeX\else
  % xetex/luatex font selection
\fi
% Use upquote if available, for straight quotes in verbatim environments
\IfFileExists{upquote.sty}{\usepackage{upquote}}{}
\IfFileExists{microtype.sty}{% use microtype if available
  \usepackage[]{microtype}
  \UseMicrotypeSet[protrusion]{basicmath} % disable protrusion for tt fonts
}{}
\makeatletter
\@ifundefined{KOMAClassName}{% if non-KOMA class
  \IfFileExists{parskip.sty}{%
    \usepackage{parskip}
  }{% else
    \setlength{\parindent}{0pt}
    \setlength{\parskip}{6pt plus 2pt minus 1pt}}
}{% if KOMA class
  \KOMAoptions{parskip=half}}
\makeatother
\usepackage{xcolor}
\usepackage{longtable,booktabs,array}
\usepackage{calc} % for calculating minipage widths
% Correct order of tables after \paragraph or \subparagraph
\usepackage{etoolbox}
\makeatletter
\patchcmd\longtable{\par}{\if@noskipsec\mbox{}\fi\par}{}{}
\makeatother
% Allow footnotes in longtable head/foot
\IfFileExists{footnotehyper.sty}{\usepackage{footnotehyper}}{\usepackage{footnote}}
\makesavenoteenv{longtable}
\usepackage{graphicx}
\makeatletter
\def\maxwidth{\ifdim\Gin@nat@width>\linewidth\linewidth\else\Gin@nat@width\fi}
\def\maxheight{\ifdim\Gin@nat@height>\textheight\textheight\else\Gin@nat@height\fi}
\makeatother
% Scale images if necessary, so that they will not overflow the page
% margins by default, and it is still possible to overwrite the defaults
% using explicit options in \includegraphics[width, height, ...]{}
\setkeys{Gin}{width=\maxwidth,height=\maxheight,keepaspectratio}
% Set default figure placement to htbp
\makeatletter
\def\fps@figure{htbp}
\makeatother
\setlength{\emergencystretch}{3em} % prevent overfull lines
\providecommand{\tightlist}{%
  \setlength{\itemsep}{0pt}\setlength{\parskip}{0pt}}
\setcounter{secnumdepth}{5}
\usepackage{booktabs}
\usepackage{amsthm}
\makeatletter
\def\thm@space@setup{%
  \thm@preskip=8pt plus 2pt minus 4pt
  \thm@postskip=\thm@preskip
}
\makeatother
\ifLuaTeX
  \usepackage{selnolig}  % disable illegal ligatures
\fi
\usepackage[]{natbib}
\bibliographystyle{apalike}
\IfFileExists{bookmark.sty}{\usepackage{bookmark}}{\usepackage{hyperref}}
\IfFileExists{xurl.sty}{\usepackage{xurl}}{} % add URL line breaks if available
\urlstyle{same}
\hypersetup{
  pdftitle={AI and the Evolving Competitive Landscape: Navigating Regulatory and Strategic Shifts},
  pdfauthor={Manav Adwani, Sohil Apte, Lauren Brown, Anson Lee, Xuexin Li},
  hidelinks,
  pdfcreator={LaTeX via pandoc}}

\title{AI and the Evolving Competitive Landscape: Navigating Regulatory and Strategic Shifts}
\author{Manav Adwani, Sohil Apte, Lauren Brown, Anson Lee, Xuexin Li}
\date{2023-08-19}

\begin{document}
\maketitle

{
\setcounter{tocdepth}{1}
\tableofcontents
}
\hypertarget{introduction}{%
\chapter{Introduction}\label{introduction}}

From assisting with strategy to generating content, artificial intelligence (AI) is already influencing how businesses operate. The consulting firm Accenture projects an average profitability boost of 38\% by 2035 across sixteen major industries. Additionally, AI has led to increased productivity and reduced costs, particularly in analytics, product development, and customer service. This has led many businesses to adopt AI with large firms like Google and Microsoft potentially investing billions in the technology. It is apparent that AI will provide a new arena for businesses to compete in.

Yet, the excitement over the benefits of AI has yielded challenges for businesses and regulators. For businesses, there are questions on what duties to assign AI. One of the causes of the ongoing actor and writer strikes is concerns over AI replacing human workers. Technology contractors in the Republic of Kenya are suing Meta for being exposed to psychologically damaging content while moderating its generative AI systems. Additionally, the exact information necessary for AI to arrive at optimal solutions is not known. Recent studies have shown there are limits to the effectiveness of pricing algorithms and that data reaches a point of diminishing returns. AI has also proven to be difficult to regulate; governments and institutions face the challenge of controlling something without knowing its full potential. The laws they establish will inevitably provide another dimension to business competition.

As businesses adopt AI, it is crucial to consider both the benefits and challenges to best predict its impact on the competitive landscape. To what degree will businesses rely on these systems and how will their competitors respond? What industries are likely to see the most meaningful change in operations? And what does this mean for the workers within those industries? Acknowledging these questions is the key to developing a business environment that utilizes AI in an efficient, fair, and ethical manner.

\hypertarget{about-us}{%
\chapter{About Us}\label{about-us}}

\hypertarget{mbangels}{%
\section{MBAngels}\label{mbangels}}

We are a team of Master of Business Analytics (MBAn) students with the goal of understanding how the new age of generative artificial intelligence (AI) will impact the world, and specifcally, impact our cohort as analytics students. By combining our wide array of backrounds, diverse skillsets, and range of persepctives, we hope to compile a robust report that captures the impact AI will have on all levels of industry.

\hypertarget{manav-adwani}{%
\section{Manav Adwani}\label{manav-adwani}}

\textbf{Background:}
Manav studied commerce in his undergrad and comes from a family business background. After completing a relatively generic degree for his undergrad, he now wants to explore the true power of business when data analytics is applied to it. This would include learning more about the strategy side of analytics.

\textbf{Experience:}
While he is enthusiastic about learning the technical aspect, he hopes to provide an in depth business perspective to the team. The team can benefit from his experience in EY and his knowledge about setting up and running a successful international business as an entrepreneur.

\hypertarget{sohil-apte}{%
\section{Sohil Apte}\label{sohil-apte}}

\textbf{Background:}
Sohil is originally from Canton Ma, and graduated from the University of Michigan in 2023. With a BS in Computer Science and a concentration on artificial intelligence and machine learning, Sohil has a robust theoretical perspective of AI, and an understanding of the intricacies of machine learning applications.

\textbf{Experience:}
As a former Computer Science student with practical experience as a Software/Machine Learning Engineer at a startup, his knowledge of machine learning model development and deployment, along with its business implications, offers insights into the real-world competitive impact of AI technologies.

\hypertarget{lauren-brown}{%
\section{Lauren Brown}\label{lauren-brown}}

\textbf{Background:}
Lauren is from Brighton, Michigan and graduated from the school of LSA at the University of Michigan in 2023. Being a Psychology major, Lauren is interested in society's reaction to AI and its outcomes.

\textbf{Experience:}
Lauren has experience in marketing, and has developed a passion for AI regulation as she recognized the potential for AI's powerful impact on the industry alongside the need for ethical guidelines to protect consumers and ensure fair practices.

\hypertarget{anson-lee}{%
\section{Anson Lee}\label{anson-lee}}

\textbf{Background:}
Anson Lee is originally from Poughkeepsie, New York. He graduated with a BBA from the Ross School of Business in 2023.

\textbf{Experience:}
His experience includes working in the operations departments of Goldman Sachs and Toyoda Gosei North America. He is also an avid cinephile and skier. Anson Lee is primarily concerned with how AI will be regulated to protect workers and industries.

\hypertarget{xuexin-li}{%
\section{Xuexin Li}\label{xuexin-li}}

\textbf{Background:}
Xuexin is originally from Beijing, China. Xuexin studied in Business Management and Economics at UC Santa Cruz during her undergraduate years.

\textbf{Experience:} Xuexin has immersed herself in the Digital Marketing industry in China after graduation. Her experience has sparked a deep interest in exploring AI's transformative potential in reshaping marketing strategies and consumer behavior.

Our combined skills and shared passion for the topic make us uniquely equipped to tackle the questions surrounding AI and the impacts it will have on comeptition and strategy. We are excited to share our research and insights in this paper.

\hypertarget{the-rise-of-ai-in-business}{%
\chapter{The Rise of AI in Business}\label{the-rise-of-ai-in-business}}

\hypertarget{xuexin-likes-the-simpsons.}{%
\section{Xuexin likes the Simpsons.}\label{xuexin-likes-the-simpsons.}}

\hypertarget{manavs-backk}{%
\section{Manav's backk!!}\label{manavs-backk}}

Please let this be a formal apology to Sohil and everyone else in the group for not passing it off to him for the Bobby Madamanchi quote in our group presentation.

\hypertarget{the-influence-of-ai-on-business-operations}{%
\section{The Influence of AI on Business Operations}\label{the-influence-of-ai-on-business-operations}}

\hypertarget{the-potential-of-ai-productivity-and-profitability}{%
\section{The Potential of AI: Productivity and Profitability}\label{the-potential-of-ai-productivity-and-profitability}}

\hypertarget{major-players-ai-investments-by-google-and-microsoft}{%
\section{Major Players: AI Investments by Google and Microsoft}\label{major-players-ai-investments-by-google-and-microsoft}}

\hypertarget{regulation-of-ai}{%
\chapter{Regulation of AI}\label{regulation-of-ai}}

AI technology has been making major advances with AI systems, such as ChatGPT, being used by over 40\% of college students in the United States (Balderson, 2023). Companies and organizations all around the world are increasing their use of AI to gain a competitive advantage in their respective industries. This race of AI technology is putting focus on what regulation needs to be developed to control these advances. AI is becoming a core part of political discourse, and big names in the tech industry such as Steve Wozniak, Elon Musk, and politicians such as Andrew Yang, have signed an open letter asking for a moratorium on AI experimentation for regulation development (Pause Giant AI experiments: An open letter, 2023). While this open letter has not led to any action, it does bring attention to the fact that governments will have to focus a large part of their future legislation on regulation. AI systems and regulations are global and inter-industry issues. There are so many different pieces, in different industries, and they're all moving at different paces, which makes it difficult to legislate.

\hypertarget{current-regulation}{%
\section{Current regulation}\label{current-regulation}}

Governments around the world have only scratched the surface on legislating AI regulation. Most countries, like the United States, are still debating what parts of AI should be regulated, and how much regulation is appropriate. There are a few exceptions to this trend. The European Parliament has recently taken its first step to legal regulation by passing a draft law named the AI Act. This act will focus on limiting the most high-risk parts of AI technology, such as facial recognition. It will also force upcoming AI systems to be more open about the data they are using. While this is only a draft law, the final version of this law is expected to be passed by the end of 2023, making it one of the first large steps toward government-enforced AI regulation (Satariano, 2023).

\hypertarget{proposed-regulation}{%
\section{Proposed regulation}\label{proposed-regulation}}

AI regulations are slowly starting to be put into effect, but there are still so many questions on how it will work. The CEO of OpenAI, Sam Altman, wrote a testimony to the Senate this May asking that they start strongly considering regulating artificial intelligence (Hendrix, 2023). He proposed solutions such as licensing companies to use AI, or creating a government agency strictly responsible for AI regulation. Altman's solutions were questioned by some, though, as there are still many disagreements about what the best way to legislate AI should be. While these solutions seem good on paper, the answer to how to regulate AI is not that simple. Officials are predicting that AI regulation agencies have the potential of being compromised. Having only one small sector of government responsible for such a broad and important issue could lead to bias. Experts have recommended that AI needs to be diversely regulated by a combination of different agencies and organizations. They have said that regulation should be created by a collaboration from academia, policy experts, industry experts, and even international agencies (Susarla, 2023).

New strategies for AI regulation are being proposed regularly by CEOs and other big names in the tech field. AI technology is progressing rapidly and there is an understanding of the need to regulate it before it gets bigger, faster, and stronger. While most tech officials are in agreement about the importance of regulation, government officials can often be harder to come by. The median age of the U.S. The House of Representatives is 58 years old, and the median age is 65 years old in the Senate. The United States lawmaker demographic is much older than the mean age in America (38 years old) (Blazina and DeSilver, 2023). Their age could mean they have lesser knowledge or insight into these new technological advancements than young, up-and-coming workers in the tech industry do. The lawmakers' lack of understanding of artificial intelligence applications can lead to more debate than needed on what type of regulation should be established.

These are prime years to establish effective regulations on artificial intelligence and how it is used. Setting regulations now, when AI is not overwhelmingly powerful, could be key to getting a grasp on this technology and its future that we cannot predict. Some places, like the European Union, are taking their first steps toward government action, but others, like the United States, have barely scratched the surface. While companies are already beginning to integrate AI regulation into their business, not every organization will follow suit without government intervention. AI Regulation should be expected to be a hot topic in government and politics in the near future since AI will continue to progress and the demand for regulation will increase with it.

\hypertarget{regulation-as-a-competitive-advantage}{%
\section{Regulation as a competitive advantage}\label{regulation-as-a-competitive-advantage}}

AI use within companies is becoming increasingly normalized in numerous industries - leading to a higher mistrust in AI and AI regulation among customers and employees of these companies. So much of what artificial intelligence is can be difficult for the average individual to grasp. For some people, the first thing they think of when they hear ``AI'' is horror stories of technology taking over the world. Acknowledging these individuals and understanding their views is a step that some companies have already started to take.

\hypertarget{pros}{%
\subsection{Pros}\label{pros}}

Responsible AI is a process that certain companies have acquired to regulate their own use of AI. It is the process of developing AI systems that minimizes biases, enforces data security, ensures transparency within companies, and creates opportunities for employees to speak out on their concerns on AI use. Companies using this form of AI regulation have actually found that it can be used as a competitive advantage. Currently, 36\% of organizations have said they believe it will create opportunities for competitive differentiation (Eitel-Porter, 2023). The ability to deliver trustworthy AI systems that are regulation-ready can help a company attract new customers because they are using AI safely and are being transparent about its use. While this is a step in the right direction, AI regulation needs to be supported by governments to truly make a difference.

\hypertarget{cons}{%
\subsection{Cons}\label{cons}}

While there are obvious benefits to regulating the use of AI, there are also a number of setbacks that could have a negative effect. Limiting its capabilities or room for growth could lead to artificial intelligence not having the ability or flexibility to adapt or learn in positive ways. Strict regulations could restrain creativity and innovation, and lessen the amount of risks taken by companies. This also could become a problem in relation to international affairs. Companies within countries that set stricter regulations will suffer in comparison to companies rooted in countries that have more flexible AI control (Balderson, 2023).

\hypertarget{the-value-of-a-ross-mban}{%
\chapter{The Value of a Ross MBAn}\label{the-value-of-a-ross-mban}}

\hypertarget{blending-technical-skills-with-business-skills}{%
\section{Blending technical skills with business skills}\label{blending-technical-skills-with-business-skills}}

As AI changes the business landscape, Master of Business Analytics (MBAn) students, from renowned institutions like the Ross School of Business, will emerge as key players. The MBAn curriculum prepares students with both the technical skills and business acumen to lead in AI-driven business domains. Executives are finding themselves facing a set of realities including untapped opportunity and existential risks (Ransbotham, 2019). As businesses invest in AI-driven operations, students with a strong understanding of technology, strategy, and leadership will be in high demand for executive roles with AI decision making.

\hypertarget{meeting-the-demand-for-ai-proficient-leaders-and-strategists}{%
\section{Meeting the demand for AI-proficient leaders and strategists}\label{meeting-the-demand-for-ai-proficient-leaders-and-strategists}}

The market is craving strong decision-makers with knowledge bases in AI. Although there is a current surplus of technical leaders like software engineers and data scientists, the biggest challenge is finding people with business expertise to contribute to AI strategy. After all, companies cannot solve strategy problems with AI without people who truly understand the underlying business goals at hand. (Yuval, 2023)

\hypertarget{the-ai-landscape-in-corporate-strategy}{%
\section{The AI Landscape in Corporate Strategy}\label{the-ai-landscape-in-corporate-strategy}}

In the current AI landscape, success is derived from harnessing its vast potential to enhance efficiency, foster innovation, and secure a competitive advantage. These advantages can only be realized if executives adopt and scale AI practices in their company in an effective manner.

\hypertarget{cultural-and-structural-obstacles-to-ai-integration}{%
\section{Cultural and Structural Obstacles to AI Integration}\label{cultural-and-structural-obstacles-to-ai-integration}}

The integration of artificial intelligence into the corporate ecosystem is a compelling value proposition for any company. However, only ``8\% of firms engage in core practices that support widespread adoption of AI and advanced analytics (Harvard Business Review, 2020)''. Just ``12\% of the executives and senior executives have included AI initiatives in their corporate strategies.'' (Mostafa, 2023) Even with so much to gain from the integration of AI, many organizations find themselves faced with cultural and structural barriers.

\hypertarget{misperceptions}{%
\subsection{Misperceptions}\label{misperceptions}}

The polarizing nature of the discourse surrounding AI has led to hesitancy in its implementation and adoption. When executives think about automating strategy, many are looking too far ahead, focusing on how AI would completely replace business leaders and their decision-making. Shifting their focus to the use of AI as ``building blocks'' of their corporate strategy could significantly improve their business outcomes (Yuval, 2023). Another misconception is the expectation of instant rewards from AI as simply a plug-and-play solution. Even when AI is adopted as a business strategy, it has a constrained use case, which limits its versatility and curtails its ultimate potential (Harvard Business Review, 2020).

\hypertarget{cultural-reluctance}{%
\subsection{Cultural Reluctance}\label{cultural-reluctance}}

The nuances of human behavior play critical roles in AI's corporate integration. Humans are resistant to change by nature, especially when it disrupts long-established tried and true norms. Changes in a business setting are no exception. Many established organizations operate on top-down decision-making. Transitioning to a model where decisions are reinforced by AI represents an upending structural and cultural shift from intuition-based CEO decision-making to a flattened organizational structure backed by data-reinforced conclusions. (Mostafa, 2023).

Overlaying these challenges is the undercurrent of the fear of the unknown. There is anxiety over job losses and reduced human significance. McKinsey highlights the issue directly: ``The big challenge is finding strategists to contribute to the AI effort. You are asking people to get involved in an initiative that may make their jobs less important (Yuval, 2023).''

\hypertarget{overcoming-barriers-to-entry}{%
\section{Overcoming Barriers to Entry}\label{overcoming-barriers-to-entry}}

Incorporating artificial intelligence into corporate strategy requires a holistic approach. There are key considerations necessary to ensure a smooth transition into an AI-integrated business.

\hypertarget{enhanced-human-ai-interactions}{%
\subsection{Enhanced Human-AI Interactions}\label{enhanced-human-ai-interactions}}

The incorporation of AI into the business world is a symbiotic relationship that demands continual human feedback for improvement (Degnan, 2023). Companies must also embrace diverse teams with varying skills and perspectives to ensure that AI solutions cater to a wide span of organizational needs. After all, the only way to maximize the potential of a data-driven technology like AI is to accommodate it with many diverse inputs (Harvard Business Review, 2020).

\hypertarget{culture-evolution}{%
\subsection{Culture Evolution}\label{culture-evolution}}

Effective implementation of AI in businesses means transitioning the corporate culture to embrace AI rather than just accepting it (Degnan, 2023). In addition to a deliberate reception of AI, an organization's attitude towards its deployment should be dynamic and iterative. By valuing continuous feedback and experimentation, organizations can develop robust AI solutions that can operate at scale. Notably, almost 90\% of companies successful in scaling AI practices allocated more than half of their analytics budgets to adoption-centric activities - emphasizing the importance of company-wide cultural shifts (Harvard Business Review, 2020).

\hypertarget{alignment-with-core-objectives}{%
\subsection{Alignment with Core Objectives}\label{alignment-with-core-objectives}}

AI should resonate with a company's mission and objectives. For example, Southwest Airlines emphasizes low-cost airfare. Their AI strategies amplify those core beliefs and provide customer value consistent with the company's values (Degnan, 2023).

\hypertarget{consistent-evaluation}{%
\subsection{Consistent Evaluation}\label{consistent-evaluation}}

Successful AI integration requires consistent performance monitoring. Companies should develop a robust set of Key Performance Indicators (KPIs) to measure how closely the AIs contributions continue to align with core objectives and provide tangible value (Degnan, 2023).

By addressing these tenets of AI strategy, organizations can pave the way to seamless integration and success - with MBAn students leading the way.

\hypertarget{conclusion}{%
\chapter{Conclusion}\label{conclusion}}

The MBAn student body and the business community as a whole stand on the precipice of a new age of commerce. The benefits provided by AI have allowed firms across numerous industries to improve on their processes both internal and external. According to a global survey by McKinsey, a significant portion of executives from companies that have adopted AI report an increase in revenue in the business areas it influenced. Specifically, 63\% of respondents claimed a revenue boost from AI integration. AI's potential to optimize company-wide operations also translates into significant cost savings. McKinsey's report highlights that 44\% of the executives observed a reduction in costs due to the adoption of AI in their business operations (McKinsey Company, 2019). Business students should consider AI's current impact and its future so they may thrive as tomorrow's leaders. Yet, this is a task easier said than done; with so many industries applying AI in so many ways, it is difficult to understand such a revolutionary technology via a holistic analysis. This report has mainly focused on front-stage elements in a few key industries that individuals are likely to interact with as workers and consumers. It is also important to acknowledge that students are not the only ones struggling to understand the impact of AI. There is an immense need for regulation, and it falls on business leaders informed on the issue to guide governments toward effective policies. This is especially important considering the conflict between managers, workers, and technology that is ongoing. MBAn students therefore face a dual responsibility. The first is to research AI applications to build more efficient businesses. The second is to develop systems of control for AI so that it does not have an adverse impact on consumers or businesses. The journey of integrating AI into the business landscape, while promising transformative benefits, has many challenges. Understanding and addressing these obstacles is necessary for any future executive looking to harness AI's full potential in their organization.

\hypertarget{work-cited}{%
\chapter{Work Cited}\label{work-cited}}

ADA editorial. Alibaba's path to ai dominance. AI, Data \& Analytics Network.October 6, 2022, \url{https://www.aidataanalytics.network/data-science-ai/articles/alibabas-path-to-ai-dominance}

Balderson, Keelan. ``27 AI in Education Statistics You Should Know.'' MSPoweruser, 23 July 2023, mspoweruser.com/ai-in-education-statistics/\#:\textasciitilde:text=couple\%20of\%20years.-,43\%25\%20of\%20US\%20college\%20students\%20admit\%20to\%20using\%20AI\%20tools,save\%20over\%2034\%2C700\%20failing\%20students.

Chakravarty, Ananda. An Intelligence-Driven Theory of Retail. IDC Perspective, November, 2022.
Den Hamer, Pieter, et al.~Apply AI Industries. Gartner, May 23, 2023.

Eitel-Porter, Ray. ``From AI Compliance to Competitive Advantage.'' Accenture, Accenture, 6 July 2023, www.accenture.com/us-en/insights/artificial-intelligence/ai-compliance-competitive-advantage.

Global Artificial Intelligence (AI) Market in Retail Sector Market 2022-2026. Infiniti Research Limited, 2022.
Hancock, B., Schaninger, B., \& Yee, L. (2023, June 5). Generative AI and the future of HR. McKinsey \& Company. \url{https://www.mckinsey.com/capabilities/people-and-organizational-performance/our-insights/generative-ai-and-the-future-of-hr}

Hetu, Robert. Infographic: Artificial Intelligence Use-Case Prism for Short Life Cycle Retail. Gartner, June 15, 2022.

Hyams, Joe. ``Will Regulating AI Hinder Innovation?'' Trullion, 10 Aug.~2023, trullion.com/blog/ai-regulation/\#:\textasciitilde:text=AI\%20systems\%20need\%20the\%20flexibility,deployment\%20of\%20beneficial\%20AI\%20applications.

Iansiti, Marco. The Value of Data and Its Impact on Competition. Harvard Business School, 2021.
Sanchez-Cartas, J. Manuel, and Evangelos Katsamakas. Artificial Intelligence, Algorithmic Competition, and Market Structures. IEEE Access, January 18, 2022.

Taylor, T. (2023, July 10). The future of AI in customer service {[}data + expert-backed predictions{]}. HubSpot Blog. \url{https://blog.hubspot.com/service/future-of-ai-in-customer-service\#}:\textasciitilde:text=AI\%20will\%20 have\%20more\%20 enhanced\%20 predictive\%20 capabilities.text=This\%20information\%20can\%20prompt\%20customer,optimal\%20point\%20of\%20their\%20journey.

Toonkell, Jessica and Amol Sharma. Hollywood's Fight: How Much AI Is Too Much?. The Wall Street Journal, July 31, 2023.

V K, A. F.. Top 5 businesses that AI transformed. Spiceworks. February 10, 2022, \url{https://www.spiceworks.com/tech/artificial-intelligence/articles/businesses-that-ai-transformed/}
Witcher, Brendan, et al.~Prediction 2023: Retail. Forrest Research, Inc., October 26, 2022.

Trend Opportunity Profile Series-Retail. Frost \& Sullivan, June 2023.

Wolinsky, J. (2023, June 28). Ai's transformative impact on future marketing strategies. CMSWire.com. \url{https://www.cmswire.com/digital-marketing/whats-coming-next-decade-for-ai-in-marketing/\#}
Sam Ransbotham, Shervin Khodabandeh. ``Winning with Ai.'' MIT Sloan Management Review, 15 Oct.~2019, sloanreview.mit.edu/projects/winning-with-ai/.

Atsmon, Yuval. ``Artificial Intelligence in Strategy.'' McKinsey \& Company, McKinsey \& Company, 11 Jan.~2023, www.mckinsey.com/capabilities/strategy-and-corporate-finance/our-insights/artificial-intelligence-in-strategy.

``Global AI Survey: Ai Proves Its Worth, but Few Scale Impact.'' McKinsey \& Company, McKinsey \& Company, 22 Nov.~2019, www.mckinsey.com/featured-insights/artificial-intelligence/global-ai-survey-ai-proves-its-worth-but-few-scale-impact.

``Building the AI-Powered Organization.'' Harvard Business Review, 1 June 2020, hbr.org/2019/07/building-the-ai-powered-organization.

Sayyadi, Mostafa, and Luca Collina. ``How to Adapt to AI in Strategic Management.'' California Management Review, 5 June 2023, cmr.berkeley.edu/2023/06/how-to-adapt-to-ai-in-strategic-management/.

Hendrix, Justin. ``Transcript: Senate Judiciary Subcommittee Hearing on Oversight of Ai.'' Tech Policy Press, 17 May 2023, techpolicy.press/transcript-senate-judiciary-subcommittee-hearing-on-oversight-of-ai/.

Solon, Olivia. How A Book About Flies Came To Be Priced \$24 Million On Amazon. WIRED, April 27, 2011.
Susarla, Anjana. ``What Would AI Regulation Look Like?'' Gizmodo, Gizmodo, 4 June 2023, gizmodo.com/chatgpt-ai-what-would-ai-regulation-look-like-altman-1850501332.

Blazina, Carrie, and Drew DeSilver. ``House Gets Younger, Senate Gets Older: A Look at the Age and Generation of Lawmakers in the 118th Congress.'' Pew Research Center, Pew Research Center, 31 Jan.~2023, www.pewresearch.org/short-reads/2023/01/30/house-gets-younger-senate-gets-older-a-look-at-the-age-and-generation-of-lawmakers-in-the-118th-congress/.

``Pause Giant AI Experiments: An Open Letter.'' Future of Life Institute, 19 Aug.~2023, futureoflife.org/open-letter/pause-giant-ai-experiments/.

Satariano, Adam. ``Europeans Take a Major Step toward Regulating A.I.'' The New York Times, The New York Times, 14 June 2023, www.nytimes.com/2023/06/14/technology/europe-ai-regulation.html.

Strategy 5 Ways to an Effective AI Corporate - Cmr.Berkeley.Edu, cmr.berkeley.edu/assets/documents/pdf/2020-02-effective-ai-corporate-strategy.pdf. Accessed 20 Aug.~2023.
Sutter John, Amazon seller lists book at \$23,698,655.93 -- plus shipping. CNN, April 25, 2011.

  \bibliography{book.bib,packages.bib}

\end{document}
